\section{Diskussion}
\label{sec:Diskussion}

Die Durchlasskurve des Bandpasses scheint ausreichend scharf zu sein, um Störungen zuverlässig zu beseitigen. Die Spannungsmesswerte
zeigen nur eine kleine Streuung auf (vergl. $\chi$-Fehler in Tabelle~\ref{tab:results}).

Die Unterschiede zwischen den beiden Messmethoden sind jedoch gravierend: Während die durch Abgleichen der Brücke gewonnenen Messwerte gut zu den Vorhersagen der Theorie passen, weichen die Spannungsmesswerte sehr stark ab. Dies liegt vermutlich an der Spannungsquelle: Diese musste nach dem Vermessen der Güte gewechselt werden. Die Frequenz ließ sich an der zweiten Quelle nur grob einstellen, sodass wahrscheinlich eine Frequenz gewählt wurde, die nicht mehr in der Nähe des Peaks der Durchlasskurve liegt.
%und so maximal deutlich weniger als die \SI{1}{\volt} der Speisespannung gemessen werden kann.
Dies würde alle Messwerte der Spannungsmessmethode stark verfälschen, die durch abgleichen der Brücke gewonnene Messwerte jedoch unberührt lassen.

Während der Messung war zu bemerken, dass ein Drehen der Probe in der Spule eine dauerhafte Spannungsänderung in der Brücke
bewirkte. Die Proben waren scheinbar nicht isotrop.

Eine weitere Fehlerquelle liegt in der Abschätzung des Probendurchmessers. Die Länge des in der Spule befindlichen Teils
der Probe ist von außen nicht klar zu beurteilen.

\newpage
\section{Literaturangabe}
\label{sec:Literatur}

Bilder und Daten aus dem Skript zu \emph{Messung der Suszeptibilität paramagnetischer Substanzen}, Versuch 606, TU Dortmund, abrufbar auf:\\
\url{http://129.217.224.2/HOMEPAGE/PHYSIKER/BACHELOR/AP/SKRIPT/V606.pdf}\\(Stand 30.05.16)\par


NIST, \emph{Theoriewerte der physikalischen Konstanten}, abrufbar auf:\\
\url{http://physics.nist.gov/cuu/index.html}\\
(Stand 16.05.16)\par

% NIST, \emph{Elektronen Masse}, abrufbar auf:\\
% \url{http://physics.nist.gov/cgi-bin/cuu/Value?me}
% (Stand 16.05.16)\par

% NIST, \emph{Plancksches Wirkungsquantum}, abrufbar auf:\\
% \url{http://physics.nist.gov/cgi-bin/cuu/Value?h|search_for=planck}
% (Stand 16.05.16)\par

% NIST, \emph{Boltzmannkonstante}, abrufbar auf:\\
% \url{http://physics.nist.gov/cgi-bin/cuu/Value?k|search_for=boltzmann}
% (Stand 16.05.16)\par
